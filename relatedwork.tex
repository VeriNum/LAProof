\section{Related Work}\label{sec:relatedwork} The challenge of finding
practically useful methods for guaranteeing the correctness and accuracy of
numerical programs is an old one. While a variety of approaches have been
successfully explored, formal static analysis methods have historically been the
least prominent. We concentrate here on the closest related work, which we
believe falls into three fairly distinct categories: formal tools for
floating-point error analysis, formalizations of numerical linear algebra, and
end-to-end machine checked proofs.

\paragraph*{Formal tools for floating-point error analysis} There are several
tools that perform rounding error analysis and generate machine-checkable proof
certificates with varying levels of automation: Gappa~\cite{gappa2008} is
implemented in C++ and produces proof scripts that can be checked in Coq;
PRECiSA~\cite{precisa2017} is implemented in Haskell and C and generates proofs
in the PVS~\cite{pvs} proof assistant, FPTaylor~\cite{fptaylor2018} is
implemented in OCaml and can produce proof certificates in HOL Light;
VCFloat~\cite{vcfloat1,vcfloat2} is implemented in Coq; and
Daisy~\cite{daisy2018} is implemented in Scala, producing proof scripts
that can be checked by Coq and HOL4~\cite{flover2018}. In general, these
tools focus on automatically obtaining tight forward error bounds for arithmetic
expressions in a given precision---that is, \emph{straight-line loop bodies.}
The goal of the LAProof library is fundamentally different: to provide formal
proofs of widely accepted mixed forward-backward error bounds for standard
\emph{algorithms} that can be used modularly in larger verification efforts.

\paragraph*{Formalizations of numerical linear algebra} With regard to the basic
linear algebra operations, Roux~\cite{roux_15} formally proved forward
error bounds in Coq for finite precision inner product and summation, and used 
these bounds to derive a formal mechanized proof of the accuracy of a finite precision
algorithm for the Cholesky decomposition. The author proves
that the formal model for floating-point arithmetic used in their
formalization satisfies the IEEE 754 binary format specified by
Flocq.

\paragraph*{End-to-end machine-checked proofs} We demonstrated the intended
functionality of the LAProof library with the verification of a C program
implementing sparse matrix-vector multiplication. Rather than serving as its own
end-to-end verification effort, the LAProof library is intended to serve as a
\emph{proof layer} between the verification of application software and programs
implementing operations defined by BLAS. There are a few end-to-end
machine-checked proofs of numerical programs in the literature that we believe
could have benefited from modular, verified building blocks like those provided
by LAProof.

Boldo and co-authors developed a machine-checked Coq proof of the correctness
and accuracy of a C program implementing a second-order finite difference scheme
for solving the one-dimensional acoustic wave equation~\cite{boldo2014}.
Scaling their results to higher dimensions would require formal error bounds for
the accuracy of basic linear algebra operations. Similarly, Kellison and
co-authors developed a machine checked proof of the correctness and accuracy of
an implementation of velocity-Verlet integration of the simple harmonic
oscillator~ \cite{kellison2022}. They obtain a forward error bound for the round
off error of their method; but a mixed backward-forward error result for
matrix-vector multiplication could have produced a tighter and more general
bound.
