\section{Introduction}\label{sec:introduction}
Numerical linear algebra is widely used across computational
disciplines and is serving an increasingly important role in
emerging applications for embedded systems. The Basic Linear
Algebra Subprograms (BLAS)~\cite{blas02,blast} provide a
modular, reliable standard defining a set of the most common
linear algebra operations such as the inner product and the
matrix-vector product. The software layer implementing the
operations defined by BLAS is often highly optimized and
architecture-specific, serving as an interface between hardware
and application software. While implementations of BLAS may
differ in practice,  an implementation
should guarantee numerical accuracy
with respect to widely accepted rounding error bounds. In this
paper, we report on our development of the
Linear Algebra Proof Library (LAProof), a library of formal
proofs of rounding error analyses for basic linear algebra
operations. LAProof serves as a
modular, portable \emph{proof layer} between the verification of application software and the verification of programs
implementing operations defined by BLAS. The LAProof library
makes the following contributions:

\begin{itemize}
\item \emph{Backward and mixed backward-forward error bounds.}
Previous formal rounding error analyses have exclusively focused on forward error bounds (see related work in Section
\ref{sec:relatedwork}).  We provide backward and mixed backward-forward error
bounds. This choice is advantageous from the perspectives
of both proof engineering and numerical analysis, as it preserves the separation of rounding errors from the structural conditions of
the mathematical problem being solved by the application
software. Furthermore, forward error bounds can be derived
directly from backward and mixed backward-forward error bounds.
\item \emph{No linearization of error terms.} The rounding error associated with a sequence of operations accumulates errors as
products of terms of the form $(1+\delta_i)$, where the
magnitude of each $\delta_i$ is uniformly upper-bounded by the
unit roundoff, $u$. Typically, numerical analysts simply
linearize the product of these terms, approximating the error in $n$ operations by $nu + \mathcal{O}(u^2)$. In LAProof we avoid
such approximations, giving clients of the library access to
error analyses that fully characterize the accumulation of error in any sequence of operations.
\item \emph{Minimal assumptions and soundness.} The LAProof
library is fully developed inside of the Coq proof assistant,
and assumes only the Flocq~\cite{flocq} specification of the
IEEE 754 standard~\cite{IEEEstd} for floating-point arithmetic.
LAProof's rounding error analysis is therefore
sound with respect to the IEEE standard. Furthermore, the
error bounds provided by LAProof do not assume
the absence of underflow; and where the proofs assume the
absence
of overflow, we provide a concrete example of how this
assumption can be discharged for operations where numerical
bounds on the terms in a linear algebra expression are known
(see Section \ref{sec:sparse}).
\item \emph{Connection to sparse matrix implementation in C.}
To demonstrate that our accuracy theorems can be seamlessly 
composed with correctness proofs of programs that
use nontrivial data structures, we use LAProof in the
verification of a C program implementing sparse matrix-
vector multiply using the compressed sparse row (CSR)
format.
\end{itemize}

The remaining sections of the paper clarify the contributions of the LAProof library. \emph{Section \ref{sec:lib}} introduces the basic linear algebra operations provided by LAProof and describes their formal error bounds. \emph{Section \ref{sec:models}} explains the implementations of the core LAProof operations in the Coq proof assistant, emphasizing their soundness with respect to the IEEE standard. \emph{Section \ref{sec:sparse}} demonstrates how the LAProof library can be used to guarantee the accuracy of concrete C programs using a machine-checked correctness proof of a C function implementing CSR matrix-vector multiplication. \emph{Section \ref{sec:relatedwork}} situates the LAProof library with respect to related work, and \emph{Section \ref{sec:conclusion}} discusses the current limitations of LAProof and future work.